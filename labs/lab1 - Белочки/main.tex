\documentclass[12pt, letterpaper, twoside]{article}
\usepackage[utf8]{inputenc}
\usepackage[english, bulgarian, russian]{babel}

\usepackage{chemmacros}

\usepackage[version=4]{mhchem}
\mhchemoptions{layout=stacked}
\chemsetup{formula = mhchem}
\chemsetup{modules=all}

\usepackage{chemscheme}
\renewcommand{\schemename}{Схема}

\usepackage{chemnum}
\setchemnum{format=\bfseries}

\pagestyle{empty}

\begin{document}

\begin{center}
	\LARGE{Лабораторная работа №1}\\[0.2cm]
	\large{Воробьев Игорь Дмитриевич}
	\large{26 октября 2022}
\end{center}

\section{Реакция Майяра}
\begin{enumerate}
	\item Поместим белково-углеводную смесь на предметное стекло.
	\item Нагреем стекло на спиртовке.
\end{enumerate}
\textbf{Результат:} наблюдаем появление запаха выпечки, изменение консистенции от жидкого состояния до твердого и изменение цвета от белого до желтоватого.

\section{Биуретовая реакция (белок)}
\begin{enumerate}
	\item Наливаем в пробирку $1$ мл белкового раствора. 
	\item Наливаем в пробирку $1$ мл \ce{NaOH}.
	\item Нагреем пробирку на спиртовке.
	\item Добавляем $2$ капли \ce{Cu(SO)4}.
\end{enumerate}
\textbf{Результат:} наблюдаем изменение окраски раствора с прозрачного до бурого.

\section{Биуретовая реакция (меламиновая губка)}
\begin{enumerate}
	\item Кладем в пробирку маленький кусочек меламиновой губки. 
	\item Наливаем в пробирку $1$ мл \ce{NaOH}.
	\item Нагреем пробирку на спиртовке.
	\item Добавляем $4$ капли \ce{Cu(SO)4}.
\end{enumerate}
\textbf{Результат:} наблюдаем изменение окраски раствора с прозрачного до бледно-голубого. 


\end{document}