\documentclass[12pt, letterpaper, twoside]{article}

\usepackage[version=4]{mhchem}
\mhchemoptions{layout=stacked}

\usepackage{chemmacros}
\chemsetup{formula = mhchem}
\chemsetup{modules=all}
\usepackage{chemscheme}
\renewcommand{\schemename}{Схема}
\usepackage{chemnum}
\setchemnum{format=\bfseries}

\usepackage{graphicx}%Вставка картинок правильная
\usepackage{float}%"Плавающие" картинки
\usepackage{wrapfig}%Обтекание фигур (таблиц, картинок и прочего)


\usepackage[utf8]{inputenc}
\usepackage[english, bulgarian, russian]{babel}

\pagestyle{empty}
\begin{document}
\begin{center}
	\LARGE{Лабораторная работа №3}\\[0.2cm]
	\large{Воробьев Игорь Дмитриевич}
	\large{9 ноября 2022}
\end{center}

\section{Углеводы в качестве восстановителей}
\begin{enumerate}
	\item Помещяем $1$ мл углевовода в пробирку.
	\item Добавляем $1$ мл \ce{NaOH} в пробирку.
	\item По каплям добавляем $5$ \% \ce{Cu(SO)4} оп появления голубой взвеси.
	\item Аккуратно греем.
\end{enumerate}
\textbf{Результат:} Для \textbf{глюкозы}, \textbf{мальтозы} и \textbf{лактозы} наблюдаем смену цвета сначала до желтого, а потом и до красного. Для \textbf{сахарозы} сколько бы мы не грели, наблюдаем только почернение раствора.

\end{document}