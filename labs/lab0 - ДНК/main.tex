\documentclass[12pt, letterpaper, twoside]{article}
\usepackage[utf8]{inputenc}
\usepackage[english, bulgarian, russian]{babel}

\usepackage{chemmacros}

\usepackage[version=4]{mhchem}
\mhchemoptions{layout=stacked}
\chemsetup{formula = mhchem}
\chemsetup{modules=all}

\usepackage{chemscheme}
\renewcommand{\schemename}{Схема}

\usepackage{chemnum}
\setchemnum{format=\bfseries}

\usepackage{graphicx}%Вставка картинок правильная
\usepackage{float}%"Плавающие" картинки
\usepackage{wrapfig}%Обтекание фигур (таблиц, картинок и прочего)
\pagestyle{empty}

 % Цвета для гиперссылок
\definecolor{linkcolor}{HTML}{799B03} % цвет ссылок
\definecolor{urlcolor}{HTML}{799B03} % цвет гиперссылок


\begin{document}

\begin{center}
	\LARGE{Лабораторная работа №0}\\[0.2cm]
	\large{Воробьев Игорь Дмитриевич}
	\large{19 октября 2022}
\end{center}

\section{Выделение ДНК из банана}
\begin{enumerate}
	\item Гомогенизируем банан вместе с буферным раствором в ступке.
	\item Переносим часть гомогената в пробирку.
	\item Добавлем раствор "Fairy".
	\item Фильтруем через носок.
	\item К фильтрату добавляем холодный изопропанол.
\end{enumerate}
\textbf{Результат:} наблюдаем на поверхности раствора длинные молекулы ДНК.

\section{Измерение чистоты полученных молекул ДНК}
\begin{enumerate}
	\item Сушим всплывшие в прошлом опыте молекулы.
	\item Добавлем \ce{H2O} в пробирку к ДНК.
	\item Переливаем супернатант в кювету для спектрофотометра.
	\item Записываем значение прибора на частоте $260$ нм: 0.854
	
	\item Записываем значение прибора на частоте $280$ нм: 0.690
	
\end{enumerate}
\textbf{Результат:} Так как отношение $0.854 / 0.690 < 1.6$, то трудно ситать наш образец чистым. Но все же с помощью онлайн-калькулятора (https://www.omnicalculator.com/biology/dna-concentration), подставив значения из спектрофотометра получаем что концентрация ДНК 42,7 микрограмм на миллилитр.

\end{document}