\documentclass[12pt, letterpaper, twoside]{article}
\usepackage[utf8]{inputenc}
\usepackage[english, bulgarian, russian]{babel}

\usepackage{chemmacros}

\usepackage[version=4]{mhchem}
\mhchemoptions{layout=stacked}
\chemsetup{formula = mhchem}
\chemsetup{modules=all}

\usepackage{chemscheme}
\renewcommand{\schemename}{Схема}

\usepackage{chemnum}
\setchemnum{format=\bfseries}

\usepackage{graphicx}%Вставка картинок правильная
\usepackage{float}%"Плавающие" картинки
\usepackage{wrapfig}%Обтекание фигур (таблиц, картинок и прочего)
\pagestyle{empty}

\begin{document}

\begin{center}
	\LARGE{Лабораторная работа №2}\\[0.2cm]
	\large{Воробьев Игорь Дмитриевич}
	\large{2 ноября 2022}
\end{center}

\section{Разделение пигментов по методу Крауса}
\begin{enumerate}
	\item Гомогенизируем кленовые листья в ступке.
	\item Переносим часть гомогената в пробирку.
	\item Добавлем $1$ мл 96 \% раствора спирта.
	\item Добавляем $5$ капель бензина для зажигалок.
	\item Перемешиваем.
	\item Добавляем $3$ капли \ce{H2O}.
	\item Перешиваем.
	\item Ждем.
\end{enumerate}
\textbf{Результат:} Бензин и спирт не смешиваются, так как бензин имеет неполярные молекулы, а спирт полярные. Благодаря этому мы выдим разделение пробирки на 2 части: верхнюю с бензином и нижнюю со спиртом (бензин всплывает вверх из-за меньшей плотности). Верхня фаза окрашивается в зеленый цвет из-за хлорофилла (хлорофилл легко растворяется в бензине), а нижняя окрашнена в красный (из-за антоциана, который хорошо растворяется в спирте).

\section{Выделение солей стеариновой кислоты}
\begin{enumerate}
	\item Наливаем в пробирку $1$ мл мыльного раствора.
	\item Добавляем $1$ мл раствора \ce{CrCl3}.
	\item Перемешиваем.
\end{enumerate}
\textbf{Результат:} Стериновая кислота из мыльного раствора \ce{C17H35COOH} реагирует с \ce{CrCl3}. В результате образуется стеарат хрома \ce{[C17H35COO]3Cr}, который нерастворим в воде и легче воды. Поэтому мы наблюдаем всплытие осадка. 


\end{document}